\section{Introduction}
\label{sec:introduction}
\lhead{\thesection \space Introduction}

This document is the handover report of the Software Factory project \textit{Connected.Football}, which was worked on by students of the seventh semester of the Software Engineering / Business Informatics course of Fontys University of Applied Sciences. It is intended to be read as a guide on what the project was about, how to set up the software artefacts developed during the project and how the development could be continued. The intended audience are students from the same semester as the authors or Software Engineering / Business Informatics students with a similar level of knowledge.
\newline
The handover document first of all describes the developed product itself, which is an implemented functionality of the mobile application \textit{Connected.Football}. To do so, it is described what frameworks both the application itself and the backend communicating with it are based on, as well as how they communicate. It is furthermore detailed what functionality the project group worked on, with focus on why such a functionality was necessary to be implemented.
\newline
This is followed by a guide on how to setup the developed software artefacts. This includes directions as to where to retrieve said artefacts. Following that, it is detailed how the mobile application can be built from the retrieved source code and how the application can be tested on both an emulated Android phone and an actual Android phone, as it was done during development.
\newline
The next chapter details the work of the project group on a high level view. It is explained where most of the relevant source code can be found and how the developed artefacts interact with each other. This chapter is structured around one particular use case which is used as an example to detail the structure of the source code.
\newline
To finish off, the handover document closes with suggestions as to how to continue the development. In particular, it is explained what functionality is still missing from the implemented feature. Furthermore, depending on the type of missing functionality, it is suggested where to start or what technology or frameworks to make use of.